\documentclass{report}
\usepackage{bbold}
\usepackage{amsmath}
%\usepackage{amssymb}

\begin{document}
ENEE 381 HW \#3 \\
Jacob Besteman-Street \\
\today \\
I recently injured my wrist, which has made my already poor handwriting downright illegible.
While it recovers, I'm using \LaTeX to write up homework assignments. If that is unacceptable, let me know.

Thanks,

Jake Besteman

\begin{itemize}
  \item[1)] \textbf{Problem 9.14:} Calculate the \textit{exact} reflection and transmission coeffecients, \textit{without}
  assuming that $\mu_1 = \mu_2 = \mu_0$. Confirm that $R+T=1$. \\
  \newline
  \begin{equation}
    \label{beta}
    \beta = \frac{\mu_1v_1}{\mu_2v_2}
  \end{equation}
  \begin{equation}
    E_{0R}= \frac{1-\beta}{1+\beta}E_{0I} \text{ and } E_{0T}= \frac{2}{1+\beta}E_{0I}
  \end{equation}
  \begin{equation}
  I_I = \frac{\epsilon_1v_1}{2}|E_{0I}|^2 \text{ and }I_R = \frac{\epsilon_1v_1}{2}|E_{0R}|^2\text{ and }I_T = \frac{\epsilon_2v_2}{2}|E_{0T}|^2
  \end{equation}
Plugging the equations from (2) in for $E_{0R}$ and $E_{0T}$ in (3), we get:
\begin{equation}
R = \frac{I_R}{I_I}  = \frac{\frac{\epsilon_1v_1}{2}(\frac{1-\beta}{1+\beta})^2E_{0I}^2}{\frac{\epsilon_1v_1}{2}E_{0I}^2} = (\frac{1-\beta}{1+\beta})^2 = \frac{1-2\frac{\mu_1v_1}{\mu_2v_2}+\frac{\mu_1^2v_1^2}{\mu_2^2v_2^2}}{1+2\frac{\mu_1v_1}{\mu_2v_2}+\frac{\mu_1^2v_1^2}{\mu_2^2v_2^2}}
\end{equation}
and
\begin{equation}
T = \frac{I_T}{I_I}  = \frac{\frac{\epsilon_2v_2}{2}(\frac{2}{1+\beta})^2E_{0I}^2}{\frac{\epsilon_1v_1}{2}E_{0I}^2} = \frac{\epsilon_2v_2}{\epsilon_1v_1}(\frac{2}{1+\beta})^2 =\frac{\epsilon_2v_2}{\epsilon_1v_1} \frac{4}{1+2\frac{\mu_1v_1}{\mu_2v_2}+\frac{\mu_1^2v_1^2}{\mu_2^2v_2^2}}
\end{equation}
Show that $R+T=1$:
\begin{equation}
  R+T=\frac{1-2\frac{\mu_1v_1}{\mu_2v_2}+\frac{\mu_1^2v_1^2}{\mu_2^2v_2^2}+4\frac{\epsilon_2v_2}{\epsilon_1v_1}}{1+2\frac{\mu_1v_1}{\mu_2v_2}+\frac{\mu_1^2v_1^2}{\mu_2^2v_2^2}}
\end{equation}
For that to equal 1, we must show that
\begin{equation}
  4\frac{\epsilon_2v_2}{\epsilon_1v_1}-2\frac{\mu_1v_1}{\mu_2v_2}=2\frac{\mu_1v_1}{\mu_2v_2} \text{ or that } \frac{\epsilon_2v_2}{\epsilon_1v_1}=\frac{\mu_1v_1}{\mu_2v_2}
\end{equation}
For that we need the equations for $v_1$ and $v_2$:
\begin{equation}
  v_1 = \frac{1}{\sqrt{\mu_1\epsilon_1}} \text{ and } v_2 = \frac{1}{\sqrt{\mu_2\epsilon_2}}
\end{equation}
Then rearrange Equation (7) into:
\begin{equation}
  \frac{\mu_2\epsilon_2}{\mu_1\epsilon_1} = \frac{v_1^2}{v_2^2}
\end{equation}
And substitute for $v_1$ and $v_2$ to get:
\begin{equation}
  \frac{\mu_2\epsilon_2}{\mu_1\epsilon_1} = \frac{\frac{1}{\sqrt{\mu_1\epsilon_1}}^2}{\frac{1}{\sqrt{\mu_2\epsilon_2}}^2} = \frac{\mu_2\epsilon_2}{\mu_1\epsilon_1}
\end{equation}
This proves the first version of Equation (7), which means we can substitute into Equation (6):
\begin{equation}
  R+T=\frac{1-2\frac{\mu_1v_1}{\mu_2v_2}+\frac{\mu_1^2v_1^2}{\mu_2^2v_2^2}+4\frac{\epsilon_2v_2}{\epsilon_1v_1}}{1+2\frac{\mu_1v_1}{\mu_2v_2}+\frac{\mu_1^2v_1^2}{\mu_2^2v_2^2}} = \frac{1+2\frac{\mu_1v_1}{\mu_2v_2}+\frac{\mu_1^2v_1^2}{\mu_2^2v_2^2}}{1+2\frac{\mu_1v_1}{\mu_2v_2}+\frac{\mu_1^2v_1^2}{\mu_2^2v_2^2}} = 1
\end{equation}
\item[2)] \textbf{Problem 9.15:} Prove that for a normal incident wave, the reflected and transmitted waves have the same polarization.\\ \newline
At normal incidence, there is no component of the E or B field that is perpendicular to the boundary.
It is all parallel. Therefore, we are only concerned with the parallel boundary conditions. Specifically:
\begin{align}
(\tilde{E}_{0I}+\tilde{E}_{0R})_{x,y} & = (\tilde{E}_{0T})_{x,y}\\
\frac{1}{\mu_1}(\tilde{B}_{0I}+\tilde{B}_{0R})_{x,y} & = \frac{1}{\mu_2}(\tilde{B}_{0T})_{x,y}
\end{align}
Substituting $\tilde{B}=\frac{1}{v}\tilde{E}$ gives
\begin{equation}
  \frac{1}{\mu_1v_1}(\tilde{E}_{0I}+\tilde{E}_{0R})_{x,y}  = \frac{1}{\mu_2v_2}(\tilde{E}_{0T})_{x,y}
\end{equation}
If the polarization of the incident wave is strictly in the $\hat{x}$ direction, then the $x,y$ components of Equation (12) are:
\begin{equation}
\tilde{E}_{0I}+\tilde{E}_{0R}\cos{\theta_R} = \tilde{E}_{0T}\cos{\theta_T}
\end{equation}
\begin{equation}
  \tilde{E}_{0R}\sin{\theta_R} = \tilde{E}_{0T}\sin{\theta_T}
\end{equation}
Likewise, the $x,y$ components of Equation (14) are, respectively:
\begin{equation}
\frac{1}{\mu_1v_1}\tilde{E}_{0I}+\frac{1}{\mu_1v_1}\tilde{E}_{0R}\cos{\theta_R} = \frac{1}{\mu_2v_2}\tilde{E}_{0T}\cos{\theta_T}
\end{equation}
\begin{equation}
   \frac{1}{\mu_1v_1}\tilde{E}_{0R}\sin{\theta_R} = \frac{1}{\mu_2v_2}\tilde{E}_{0T}\sin{\theta_T}
\end{equation}

We will consider just the $y$ components:
\begin{align}
    \tilde{E}_{0R}\sin{\theta_R} & = \tilde{E}_{0T}\sin{\theta_T} \\
    \frac{\tilde{E}_{0R}}{\mu_1v_1}\sin{\theta_R} & = \frac{\tilde{E}_{0T}}{\mu_2v_2}\sin{\theta_T}
\end{align}

There are only two circumstances under which both Equation (19) and (20) can be true: \\
Either $\mu_1v_1 = \mu_2v_2$, in which case both sides of the boundary are electromagnetically identical and there \textit{is no} boundary,\\
or $sin{\theta_R} = sin{\theta_T} = 0$ which means that $\theta_R = \theta_T = 0$.
\item[3)] \textbf{Problem 9.18:} I did this one by hand, don't yet know how to do graphs in \LaTeX. It's not pretty.
\item[4)] Derive the Fresnel coefficients for an incident wave polarized in the $y$ direction, perpendicular to the incident plane.\\ \newline
This one has a lot of steps. The final answers are Equations (32) and (35) on the next page. \\ \newline
Since \~{E} is in the $y$ direction, then \~{B} will be in the incident plane $(x,z)$. Therefore there are three boundary conditions to deal with:
\begin{align}
  (\tilde{E}_{0I}+\tilde{E}_{0R})_{y} & = (\tilde{E}_{0T})_{y}\\
  (\tilde{B}_{0I}+\tilde{B}_{0R})_{z} & = (\tilde{B}_{0T})_{z}\\
  \frac{1}{\mu_1}(\tilde{B}_{0I}+\tilde{B}_{0R})_{x} & = \frac{1}{\mu_2}(\tilde{B}_{0T})_{x}
\end{align}
Which can be expanded or rearranged into:
\begin{align}
  E_{0I}+E_{0R} & = E_{0T}\\
  (B_{0I}-B_{0R})\cos{\theta_I} & = B_{0T}\cos{\theta_t}
\end{align}
The latter of which can be rewritten as
\begin{equation}
  \frac{1}{v_1} (E_{0I}-E_{0R})\cos{\theta_I}  = \frac{1}{v_2} E_{0T}\cos{\theta_t}
\end{equation}
Since $n=\frac{c}{v}$ we can rewrite that as
\begin{equation}
  n_1(E_{0I}-E_{0R})\cos{\theta_I}  = n_2 E_{0T}\cos{\theta_t}
\end{equation}
Using Equation (24), plug in $  E_{0I}+E_{0R}$ for $E_{0T}$ to get
\begin{equation}
  n_1(E_{0I}-E_{0R})\cos{\theta_I}  = n_2 (E_{0I}+E_{0R})\cos{\theta_t}
\end{equation}
Which can be rearranged into:
\begin{equation}
  \frac{E_{0R}}{E_{0I}} = \frac{n_1\cos{\theta_I}-n_2\cos{\theta_T}}{n_1\cos{\theta_I}+n_2\cos{\theta_T}}
\end{equation}
Subbing in $n=\frac{n_2}{n_1}$
\begin{equation}
  \frac{E_{0R}}{E_{0I}} = \frac{\cos{\theta_I}-n\cos{\theta_T}}{\cos{\theta_I}+n\cos{\theta_T}}
\end{equation}
To get this in terms of $\theta_I$, use the law of reflection, $n_1\sin{\theta_I}=n_2\sin{\theta_T}$ or $\sin{\theta_I}=n\sin{\theta_T}$
\begin{equation}
  \cos{\theta_T} = \sqrt{1-\sin{\theta_T}^2} = \sqrt{1-\frac{sin{\theta_I}^2}{n^2}}
\end{equation}
Substitute this into Equation (30) to get everything in terms of $n$ and $\theta_I$:
\begin{equation}
  \frac{E_{0R}}{E_{0I}} = \frac{\cos{\theta_I}-\sqrt{n^2-sin{\theta_I}^2}}{\cos{\theta_I}+\sqrt{n^2-sin{\theta_I}^2}}
\end{equation}
To find $\frac{E_{0T}}{E_{0I}}$, we solve Equation (24) for $E_{0R}$ and plug that into Equation (26) to get:
\begin{equation}
  \frac{1}{v_1} (E_{0I}-(E_{0T}-E_{0I}))\cos{\theta_I}  = \frac{1}{v_2} E_{0T}\cos{\theta_t}
\end{equation}
Once again substituting $n=\frac{1}{v}$ and rearranging terms, we get:
\begin{equation}
  \frac{E_{0T}}{E_{0I}} = \frac{2n_1\cos{\theta_I}}{n_1\cos{\theta_I}+n_2\cos{\theta_T}}
\end{equation}
Repeating the sequence of substitutions used to make Equations (30), (31), and (32) gives:
\begin{equation}
  \frac{E_{0T}}{E_{0I}} = \frac{2\cos{\theta_I}}{\cos{\theta_I}+\sqrt{n^2-sin{\theta_I}^2}}
\end{equation}
Brewster's angle occurs were $\frac{E_{0R}}{E_{0I}} = 0$. This requires the numerator of Equation (32) to be 0.
\begin{align}
  \cos{\theta_I}-\sqrt{n^2-sin{\theta_I}^2} & = 0\\
  \cos{\theta_I} & =\sqrt{n^2-sin{\theta_I}^2}\\
  \sqrt{1-sin{\theta_I}^2}& = \sqrt{n^2-sin{\theta_I}^2}\\
1-sin{\theta_I}^2 &=n^2-sin{\theta_I}^2\\
1&=n^2\\
n&=1
\end{align}
Brewster's Angle can only exist if $n=1$ or $n_1 = n_2$, in which case there is no reflection because there is no boundary. Otherwise, it is impossible.

  \end{itemize}
   \end{document}
