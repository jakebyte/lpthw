\documentclass{report}
\usepackage{bbold}
\usepackage{amsmath}

\begin{document}
ENEE 324 HW \#2 \\
Jacob Besteman-Street \\
\today \\

I recently injured my wrist, which has made my already poor handwriting downright illegible.
While it recovers, I'm using \LaTeX to write up homework assignments. If that is unacceptable, let me know.

Thanks,

Jake Besteman

\begin{enumerate}
\item A frog is trapped in a well.  In order to escape from the well the frog needs to climb up three steps.
The frog can jump high enough for one step at a time.
However, it needs to jump at a certain angle in order to land on the next step successfully.
Each time it jumps it successfully jumps to the next step with probability \textit{q}.
 Otherwise, it hits the vertical side of the stair and bounces back to where it was
 with probability 1-\textit{q} and tries again starting from the same step.  Let \textit{X}
 denote the number of jumps the frog makes before it escapes the well.

\begin{itemize}
\item[(a)] What is the probability mass function (PMF) of \textit{X}?\\ \newline
The PMF of \textit{X} is $$p_X(n) = \binom{n-1}{2}q^3(1-q)^{n-3} \text{ for} n = 3, 4, ...       $$


\item[(b)] Compute E[\textit{X}]. \\ \newline
\textit{X} is a Pascal RV, therefore its expected value $ E[X] = \frac{k}{p} $ where \textit{p} = \textit{q} and \textit{k} is the number of successes needed, 3. Therefore: $$E[X] = \frac{3}{q}$$
\item[(c)] Suppose that the frog escaped from the well after four jumps. Given this, what is the probability that the first jump was successful, i.e., the frog moved on to the first step from the bottom afterthe first jump? \\ \newline
There are 3 possible results that will get the frog out after 4 jumps: \\ \newline
\begin{tabular}{c|cccc}
Jump & 1 & 2 & 3 & 4\\
\hline
Result & Failure & Success & Success & Success \\
 & Success & Failure & Succes & Success\\
& Success & Success & Failure & Success \\
\end{tabular} \\
\newline
Intuitive Solution: Since each jump is independent, each result is equally likely. The first jump is successful for 2 of the 3 possibilities, therefore the probability that the first jump was a success is: $$P(\{\text{first jump success}\}|X = 4) =  \frac{2}{3} $$.

Proper Solution: \\
Let $ A = \{\text{First jump success}\}$ \\
We need $P(A|\{X=4\})$ \\
By the Bayes' Rule:
$$P(A|X=4) = \frac{P(X=4|A)P(A)}{P(X=4)}  $$
$P(A)=q$ and $P(X=4) = p_X(4)$, which is:
$$p_X(4) = \binom{4-1}{2}q^3(1-q)^{4-3} = 3(q^3 - q^4)$$
This leaves $P(X=4|A)$, or the probability that the frog escapes in 4 jumps if it succeeds on the first one, which is the probability that it needs 3 jumps to climb the next two steps. This can be found from recalculating $p_X(n)$ with $k=2$ and $n=3$:
$$p_X(4) = \binom{2}{1}q^2(1-q)^{1} = 2(q^2 - q^3)$$
Putting it all together,
$$P(A|X=4) = \frac{P(X=4|A)P(A)}{P(X=4)} = \frac{2(q^2 - q^3)q}{3(q^3 - q^4)} = \frac{2}{3} $$
which matches the intuitive, counting result after the \textit{q} terms cancel out.
\end{itemize}
\item Brad is looking out his window daydreaming. The number of people who pass by outside his window per hour is a Poisson random variable (rv) with
$\alpha = 24$. Let \textit{Y} be the number of people who pass by outside Brad’s window during the first hour.
%\begin{itemize}

 %\end{itemize}
\item A codeword is a sequence of 0’s and 1’s of length k. For instance, if $k = 5$, $c = 01011$ is an example
of a codeword. Each codeword is generated by an outcome of 5 coin flips as follows. The coin shows
up the head with a probability of 0.6. The \textit{k}-th bit is 1 if the outcome of the \textit{k}-th coin flip is a head.
Otherwise, it is 0, e.g. if the outcome of 5 coin flips is HTHTT, then the codeword is 10100 as before.\\ \newline
\underline{The problem does not explicitly state the value of \textit{k}, but gives 5 as an}\\ \underline{ example. I am using \textit{k}=5 to solve the problem.}
\begin{itemize}
  \item[(a)] How many codewords have 3 zeros? \\ \newline
  $$\binom{5}{3} = \frac{5!}{2!3!} = 10$$
  \item[(b)] What is the probability of generating a codeword with 2 zeros? \\ \newline
  From (c), using $k=2$,
  $$p_X(2) = \binom{5}{2}(0.4)^2(0.6)^{3} = \frac{5!}{3!2!}(0.16)(0.216) = 0.3456  $$

  \item[(c)] Let \textit{X} be the number of zeros in a codeword. What is the PMF of \textit{X}?\\ \newline
\textit{X} is the number of zeros, which is the number of coinflips that come up tails, which has probability 0.4. Therefore:
$$p_X(k) = \binom{5}{k}(0.4)^k(0.6)^{5-k}$$
\item[(d)] What is E[\textit{X}]? \\ \newline
\textit{X} is a binomial random variable, so $E[X]=n\cdot p = 5 \cdot 0.4 = 2$
\end{itemize}
\item Bob throws a dart 5 times. If he hits the bull’s eye at least 3 times, he wins \$15.
Otherwise, he needs to pay \$4. Each time he hits the bull’s eye with a probability of 0.2.
\begin{itemize}
\item[(a)] Let the rv \textit{Y} be the number of times Bob hits the bull’s eye. What is the PMF of \textit{Y}? What kindof rv is \textit{Y}?\\ \newline
\textit{Y} is a binomial random variable, with $n=5$ and $p=0.2$. Therefore the PMF of \textit{Y} is:
$$p_Y(k) = \binom{5}{k}(0.2)^k(0.8)^{5-k}$$
\item[(b)] Suppose that Bob hits the bull’s eye the first two times. What is the probability that he will win \$15?\\ \newline
If Bob has already hit the bull's eye twice, then this reduces to the chance that he will hit \textit{at least} once in the next three tries.
Another way to look at this is P(\{at least one hit\}) = 1-P(\{he misses all three remaining throws\}) or $1-0.8^3 = 0.488$

\end{itemize}
 \end{enumerate}
 \end{document}
